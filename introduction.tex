
\section{Введение}

Что бы вы смогли изменить в компании, если бы могли измерить вероятность того, что самые продуктивные сотрудники хотят покинуть компанию? 
Или из почтовой переписки между продавцами и покупателями определить, какой подход к клиенту приносит больше всего пользы?
В наше время по данным почтовых переписок, сообщениям в социальных сетях, историям телефонных звонков можно получить много информации о текущем положении дел в организации. 
В дальнейшем эти данные могут использоваться для улучшения результатов.

Мы работаем в организации, которая насчитывает тысячи сотрудников. Отлаженная коммуникация между этими людьми - залог успеха всей компании. 
Таким образом одна из наших задач - разработать метрики и программное обеспечение, которые позволят измерить эффективность коммуникации внутри компании. 
Исследование данных обанкротившихся компаний и открытых организаций, а также доступных внутренних почтовых переписок и телефонных звонков поможет нам выявить ключевые критерии высокой производительности компаний. 

Мы предполагаем, что изучение различных источников данных позволит выявить различные индикаторы эффективного сотрудничества. 
Эти индикаторы помогут менеджерам понять, как организовать работу группы и как ею руководить для решения той или иной задачи.

Например, исследования команды, участником которой был Питер Глур (Peter A.Gloor) \cite[p.~9]{peter_gloor_2016} показали, что творческие люди гораздо эффективнее, если у них есть строгий лидер. Хороший пример сотрудничества творческих людей - Wikipedia, в которой маленькие группы с сильным лидером создают качественные статьи быстрее, чем даже большее количество людей без лидера.

Также они выявили закономерность о том, что коллективы, в которых через определенное время меняется лидер, достигают больших результатов за то же время по сравнению с командами, в которых на протяжении всего времени работы лидер остается тем же самым.

Таким образом, изучив данные, определив метрики и выявив сами критерии, перед нами встанет задача - показать результаты непосредственно сотрудникам или командам. Питер Глур в своей статье называет "виртуальным зеркалом" ("virtual mirroring"\cite[p.~10]{peter_gloor_2016}) возвращение человеку его шаблона поведения,  возможно с какой-то оценкой или в сравнении с другими сотрудниками.

Также основываясь на многолетнем опыте Питера мы попытаемся разделить работу с каждым конкретным набором данных или конкретной организацией на 4 этапа:

\begin{enumerate}
	\item Определить метрики и шаблоны в коммуникации социальной сети. На этом этапе мы изучаем данные, выявляем шаблоны поведения и строим социальную сеть. Использую полученные данные, мы визуализируем социальную сеть и строим первоначальные гипотезы.
	\item Определить взаимосвязь между признаками социальной сети и успехами бизнеса. На этом этапе мы сравниваем результаты первого этапа с шаблонами, которые являются показателем лучшего взаимодействия внутри компании, которые нам еще предстоит выявить.
	\item Предоставить обратную связь сотрудникам или командам. Т.е. указать сотрудникам на те моменты, в которых они ведут себя не так продуктивно, как могли бы или вовсе непрофессионально. 
	\item Предоставить рекомендации по оптимизации коммуникаций для достижения еще большего успеха. Когда менеджер узнает, что именно в работе его команды нарушено и к чему именно нужно стремиться, он сможет перестроить процесс таким образом, чтобы увеличить производительность команды.
\end{enumerate}

 