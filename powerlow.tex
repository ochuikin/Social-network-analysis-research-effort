\section{Степенные законы распределения}

В этой части я расскажу о функциях распределения и о том, почему мы должны обратить на них внимание. Я проделаю это на примере конкретных данных обанкротившейся компании Энрон (Enron Corporation).

Энрон - компания, достигшая большого успеха в области энергетики. Однако компания обанкротилась в 2001 году в результате бухгалтерских махинаций и сокрытия реальных доходов\cite{enron_fall}. Корпус Энрон (Enron corpus) - публичный набор данных (dataset) с историей почтовых сообщений внутри компании.

С одной стороны этот корпус может представлять большой интерес для анализа, а с другой стороны он достаточно мал (примерно полмиллиона сообщений между 158 сотрудниками\cite{sims_sinitsyn_matrix_structures}).

Функции распределения сами по себе уже давно не являются чем-то новым и необычным. Очень много вещей и величин например распределены по нормальному закону. Это происходит из следствия предельной теоремы, которое говорит о том, что если у есть множество случайных распределений, то сумма этих распределений будет похожа на нормальное распределение. 

Так распределены, например, рост человека или максимальная скорость автомобилей. Если мы говорим о распределении роста человека, то оно близко к нормальному, имеется ярко выраженный узкий пик и очень быстро затухающие хвосты. 

\todo{Можно добавить картинку с распределением роста}

\includepdf[width=1\textwidth]{recources/mailes_by_people_enron.pdf}

Конечно же не все величины в природе распределены нормально. Есть множество исключений, например размер населения городов. В этом случае уже нет пика, функция всегда убывает. Это говорит о том, что случайно выбранный город с большей вероятностью окажется с небольшим населением и с очень низкой вероятностью огромным. 

\todo{Картинка с распределением населения городов}

Важно отметить, что встречаются города с очень большим населением, которое больше населения небольших городов на порядки, т.е. у этого распределения есть длинных хвост. И этот хвост затухает не экспоненциально, не быстро. 

Если это распределение построить в логарифмических координатах, то оно станет гораздо нагляднее и примет вид близкий к прямой. Такого вида распределения мы и будем называть \textit{степенными распределениями}. 

Это распределение имеет вид \cite{zhukov} $$ p(x) = Cx^{-\alpha} = \frac{C}{x^\alpha} \text{, для } x \ge x_{min} $$ 
Нормировка дает значение константы $C = (\alpha - 1) x_{min}^{\alpha - 1}$.
Таким образом в общем виде степенное распределение имеет вид $$p(x) = \frac{\alpha - 1}{x_{min}} \left( \frac{x}{x_{min}} \right)^{-\alpha}$$

Стоит отметить, что в этом законе распределения участвует две константы: $\alpha$ и $x_{min}$. Таким образом получается, что закон ведет семя немного по разному в зависимости от выбора начально значения. Это обусловлено тем, что нельзя определить закон из нуля.